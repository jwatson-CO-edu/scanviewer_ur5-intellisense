\documentclass{hw_grad}

% === RECIPES ==============================================================================================================================

% ~~ FIGURE ~~
%\begin{figure}[h]
%	\centering
%	\includegraphics[scale=0.45]{imaginativeAndGenerative}
%	\caption{Relationship between Generative and Imaginative}
%	\label{imaginativeAndGenerative}
%\end{figure}

% ~~ Answer Boxes ~~
% Inside Eq: \boxed{}
% Text:      \fbox{}
% Region:    \begin{framed} ... \end{framed}
%            - OR -
%            \fbox{% Will complain is there is not a comment at the end of the open bracket line
%            	...
%			 }

% ___ END RECIPES __________________________________________________________________________________________________________________________


\title{Presentation Notes}
\duedate{Due: 2018-12-13}
\class{CSCI-5229: Computer Graphics, Fall 2018}
\institute{University of Colorado Boulder}
\author{James Watson, 105754866}

\begin{document}
	\maketitle
	
	\begin{enumerate}
		\item Demo
		\begin{enumerate}
			\item Purpose\\
			The purpose of this project
			\item Features
			\item Process
			\begin{enumerate}
				\item Data Reading \& Processing
				\item Mesh Construction \& UV-Matching
				\item Event Loop
			\end{enumerate}
		\end{enumerate}
		\item Cover things that you did in your project that I did not talk about in class
		\begin{itemize}
			\item Processing mouse clicks in the camera frame
		\end{itemize}
		\item Cover "gotchas" - things that did not work initially, how you figured out the problem, and how you fix it
		\begin{enumerate}
			\item SDL2 function calls are all different from SDL1
			\item There are two ways to handle keyboard input in SDL2, Keyboard State \& Event Queue
			\begin{itemize}
				\item Keyboard State
				\item Event Queue
			\end{itemize}
			\item Difference between screen coordinates, orthogonal (flat) coordinates, and model (projection) coordinates
		\end{enumerate}
		\item Cover stuff related to your project that has nothing to do with Computer Graphics
		\begin{itemize}
			\item Delaunay Triangulation, courtesy of Paul Bourke
			\begin{itemize}
				\item Paul Bourke is a big name in computational geometry, and he posts a lot code online with helpful diagrams
				\item A common method to connect a collection of points in triangular mesh
				\item It is the dual of Voronoi Cells
			\end{itemize}
			\item Hierarchical Collision Detection
			\begin{enumerate}
				\item Intersection of AABB and ray
				\item Triangle-wise collision detection \\
				For each triangle
				\begin{enumerate}
					\item Construct a plane from triangle and its normal
					\item Get the intersection point between the ray and the plane
					\item Compute winding number (point-in-polygon) \\
					This works for any closed polygon, polygon can self-cross
					\begin{enumerate}
						\item New 2D reference frame with the query point at $\left( 0 , 0 \right)$ 
						\item Follow segments CW around the polygon and count $ \texttt{+} \vx $
					\end{enumerate}
				\end{enumerate}
			\end{enumerate}
		\end{itemize}
	\end{enumerate}
	
	
	
\end{document}